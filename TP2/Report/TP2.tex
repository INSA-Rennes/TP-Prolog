\documentclass[a4paper,12pt]{article}

\usepackage[latin1]{inputenc}
\usepackage[T1]{fontenc}
\usepackage[francais]{babel}
\usepackage{graphicx}
\usepackage{color}
\definecolor{grey}{rgb}{0.9,0.9,0.9}
\definecolor{teal}{rgb}{0.0,0.5,0.5}
\definecolor{violet}{rgb}{0.5,0,0.5}
\usepackage[margin=2.5cm]{geometry}\usepackage{listings}
\usepackage{listingsutf8}
\lstloadlanguages{Prolog}
\lstdefinestyle{listing}{
  language=Prolog,
  captionpos=t,
  inputencoding=utf8/latin1,
  extendedchars=true,
  numbers=left,
  numberstyle=\tiny,
  numbersep=5pt,
  breaklines=true,
  breakatwhitespace=true,
  showspaces=false,
  showstringspaces=false,
  showtabs=false,
  tabsize=2,
  basicstyle=\footnotesize\ttfamily,
  backgroundcolor=\color{grey},
  keywordstyle=\color{blue}\bfseries,
  commentstyle=\color{teal},
  identifierstyle=\color{black},
  stringstyle=\color{red},
  numberstyle=\color{violet},
}
\lstset{style=listing}

\title{TP2 - Manipulation de termes construits}
\author{\textsc{Paul Chaignon} - \textsc{Cl�ment Gautrais}}
\date{\today}

\begin{document}

\maketitle

\section{Questions}

\lstinputlisting[caption=tp2\_poker\_etud.pro]{tp2_poker_etud.pro}
\vspace{1cm}

\section{Tests}

\lstinputlisting[caption=tp2\_poker\_etud\_tests.pro]{tp2_poker_etud_tests.pro}
\vspace{0.5cm}

Les deux versions sont �quivalentes car, dans la premi�re version,
le fait que C1 et C2 sont des cartes est v�rifi� en utilisant hauteur et couleur.
Cependant la premi�re version est meilleure car l'appel � est\_carte oblige l'instanciation des param�tres.

\end{document}